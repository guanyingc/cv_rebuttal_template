\documentclass[10pt,twocolumn,letterpaper]{article}
\usepackage[rebuttal]{cvpr}

% Include other packages here, before hyperref.
\usepackage{graphicx}
\usepackage{amsmath}
\usepackage{amssymb}
\usepackage{booktabs}

\usepackage{lipsum} % Just for generating dummy text, can be removed
\usepackage{blindtext}

\usepackage[labelsep=period]{caption}
\captionsetup{font=small}
\captionsetup[table]{aboveskip=0pt}
\captionsetup[table]{belowskip=0pt}
\captionsetup[figure]{aboveskip=2pt}
\captionsetup[figure]{belowskip=0pt}


% If you comment hyperref and then uncomment it, you should delete
% egpaper.aux before re-running latex.  (Or just hit 'q' on the first latex
% run, let it finish, and you should be clear).
\usepackage[pagebackref,breaklinks,colorlinks,bookmarks=false]{hyperref}

% Support for easy cross-referencing
\usepackage[capitalize]{cleveref}
\crefname{section}{Sec.}{Secs.}
\Crefname{section}{Section}{Sections}
\Crefname{table}{Table}{Tables}
\crefname{table}{Tab.}{Tabs.}

% Change to Figure R1/ Table R1
\renewcommand{\thetable}{R\arabic{table}}
\renewcommand{\thefigure}{R\arabic{figure}}

\newcommand{\issue}[1]{\vspace{0.1em}\noindent \textcolor{blue}{\textbf{#1 \hspace{0.2em}}}}

\newcommand{\Rone}{\vspace{0.0em}\noindent \textcolor[RGB]{0, 123, 167}{\textbf{R1}\hspace{0.0em}}}
\newcommand{\Rtwo}{\vspace{0.0em}\noindent \textcolor[RGB]{204, 85, 0}{\textbf{R2}\hspace{0.0em}}}
\newcommand{\Rthree}{\vspace{0.0em}\noindent \textcolor[RGB]{0, 128, 0}{\textbf{R3}\hspace{0.0em}}}
% If you wish to avoid re-using figure, table, and equation numbers from
% the main paper, please uncomment the following and change the numbers
% appropriately.
%\setcounter{figure}{2}
%\setcounter{table}{1}
%\setcounter{equation}{2}

% If you wish to avoid re-using reference numbers from the main paper,
% please uncomment the following and change the counter for `enumiv' to
% the number of references you have in the main paper (here, 6).
%\let\oldthebibliography=\thebibliography
%\let\oldendthebibliography=\endthebibliography
%\renewenvironment{thebibliography}[1]{%
%     \oldthebibliography{#1}%
%     \setcounter{enumiv}{6}%
%}{\oldendthebibliography}


% Variables
%%%%%%%%% PAPER ID  - PLEASE UPDATE
\def\PaperID{8888} % *** Enter the CVPR Paper ID here
\def\confName{CVPR}
\def\confYear{2023}

\begin{document}

%%%%%%%%% TITLE - PLEASE UPDATE
\title{The Title of Your Paper}

\maketitle

\thispagestyle{empty}
\appendix

%%%%%%%%% BODY TEXT - ENTER YOUR RESPONSE BELOW
We thank all three reviewers for their constructive comments and appreciations of our strengths such as `the propose method is novel' (\Rone), `the ablation study is convincing' (\Rtwo), and `the method is state-of-the-art' (\Rthree).


\begin{table}[t]
\centering
\caption{Table 1.}
\label{tab:table}
    \resizebox{0.48\textwidth}{!}{
    \large
    \begin{tabular}{*{10}{c}}
        \toprule
       Data & Size &  2-Exp & 3-Exp &  4-Exp & 5-Exp &  6-Exp &  7-Exp \\
        \midrule
        A & $1280\times 720$ & 1 & 2 & 3 & 4 & 5 & 4 \\
        B & $1280\times 720$ & 1 & 2 & 3 & 4 & 5 & 4 \\
        Ours & $4096\times 2168$ & 2 & 3 & 4 & 6 & 5 & 4 \\
        \bottomrule
    \end{tabular}
    }
    \vspace{-1em}
\end{table}

\Rone: \textbf{Novelty.}
\lipsum[1]

\Rone: \textbf{Comparison with existing methods.}
\lipsum[2]

\Rtwo: \textbf{Missing details for the dataset.}
\lipsum[3]

\Rtwo: \textbf{Missing details for the method.}
\lipsum[4]

\Rtwo: \textbf{Missing references.}
We will include them. 

\begin{figure}[tb] \centering
    \includegraphics[width=0.48\textwidth,height=0.20\textwidth]{example-image}
    \caption{Image 1.} \label{fig:img1}
    \vspace{-1em}
\end{figure}

\Rthree: \textbf{Failure case.}
\lipsum[5]

\Rthree: \textbf{Visualization of the method.}
\lipsum[6]

\Rthree: \textbf{Evaluation details.}
Sed commodo posuere pede. Mauris ut est. Ut quis purus. Sed ac odio. Sed vehicula. hendrerit sem. Duis non odio. Morbi ut dui. Sed accumsan. risus eget odio. In hac habitasse platea dictumst. Pellen-esque non elit. Fusce sed justo eu urna porta tincidunt Mauris felis odio, sollicitudin sed, volutpat a, ornare ac.

\Rthree: \textbf{Choice of hyper-parameters.}
Sed commodo posuere pede. Mauris ut est. Ut quis purus. Sed ac odio. 


\Rthree: \textbf{Typos.}
We will fix them in the revision.

\end{document}

